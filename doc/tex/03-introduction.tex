\chapter*{ВВЕДЕНИЕ}
\addcontentsline{toc}{chapter}{ВВЕДЕНИЕ}


В современном мире, где время является ценным ресурсом, эффективное управление им является важной задачей как для отдельных людей, так и для организаций.
Чтобы грамотно распоряжаться своим временем, необходимо знать, сколько времени реально требуется на выполнение каждой задачи. Именно поэтому разработка приложения для учета и аудита времени, потраченного на определенную задачу, является актуальной~\cite{intro}.

Приложение, которое объединит в себе удобство канбан-досок для визуального планирования задач с интегрированным таймером для учета времени и аналитики, станет уникальным решением и поможет эффективно планировать как личные, так и рабочие задачи. Оно позволит пользователям совместно вести учет времени, что сделает его незаменимым инструментом для широкого круга пользователей, от фрилансеров до компаний, стремящихся к оптимизации своих процессов управления временем и повышению общей продуктивности.

Цель курсовой работы~---~разработка базы данных для хранения и обработки данных приложения учета и аудита времени, потраченного на рабочие и личные задачи.

Для достижения поставленной цели необходимо выполнить следующие задачи:
\begin{itemize}
	\item сформулировать описание пользователей проектируемого приложения по учету и аудиту времени, потраченного на рабочие и личные задачи для доступа к базе данных;
	\item спроектировать сущности базы данных и ограничения целостности учета и аудита времени, потраченного на рабочие и личные задачи;
	\item выбрать средства реализации и разработать базу данных и приложение;
	\item провести исследование зависимости времени выполнения запросов от количества записей в базе данных.
\end{itemize}


