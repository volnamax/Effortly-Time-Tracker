\chapter*{ЗАКЛЮЧЕНИЕ}
\addcontentsline{toc}{chapter}{ЗАКЛЮЧЕНИЕ}

Цель лабораторной работы достигнута, исследованы параллельные вычисления на основе нативных потоков.

Для достижения поставленной цели были выполнены следующие задачи:
\begin{itemize}
	\item описан алгоритм иерархической кластеризации;
	\item спроектировано программное обеспечение, реализующее алгоритм и его параллельную версию;
	\item разработано программное обеспечение, реализующее алгоритм и его
	параллельную версию;
	\item проанализированы затраты реализаций алгоритмов по времени.
\end{itemize}

В результате исследования реализуемых алгоритмов по времени выполнения можно сделать следующие выводы.
\begin{itemize}
	\item Время выполнения последовательной реализации алгоритма остается постоянным и равным 4200 мс (см. рисунок~\ref{img:threads}). На графике видно, что для небольшого числа потоков (до 8), параллельная реализация работает быстрее последовательной. Однако, с увеличением количества потоков, параллельная реализация алгоритма становится менее эффективной по времени выполнения, чем последовательная реализация алгоритма. Это можно объяснить тем, что на машине, на которой проводились исследования 12 логических ядер. При увеличении количества потоков сверх количества логических ядер, происходит конкуренция за процессорные ресурсы, что приводит к переключению контекста и, как следствие, к увеличению времени выполнения. Кроме того, на создание и управление потоками тратится дополнительное время, что также влияет на общую производительность. 
	
	\item Из таблицы~\ref{tbl:2} видно, что при малом количестве входных файлов разница во времени выполнения между последовательной реализацией и параллельной реализацией с одним дополнительным потоком, не очень заметна, но с увеличением количества файлов параллельная версия с одним дополнительным потоком становится намного эффективней по времени выполнения.
	Параллельная реализация алгоритма с одним дополнительным потоком на 38\% быстрее по сравнению с последовательной реализацией алгоритма, при обработке 50 файлов (см. таблицу~\ref{tbl:2}).
	
	\item Исходя из данных (см. таблицу~\ref{tbl:2}), оптимальное количество потоков для данной задачи находится в диапазоне от 2 до 8 (см. таблицу~\ref{tbl:3}). В этом диапазоне время выполнения минимально и значительно меньше, чем у последовательной реализации алгоритма.
\end{itemize}
