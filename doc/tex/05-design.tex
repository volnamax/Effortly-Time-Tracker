\chapter{Конструкторский раздел}

В данном разделе будут представлены требования к программному обеспечению, описание используемых типов данных и схемы реализуемых алгоритмов.

\section{Требования к программному обеспечению}

К программе предъявлен ряд требований:
\begin{itemize}
	\item должен присутствовать интерфейс для выбора действий;
	\item считывание данных должно производиться из файла;
	\item результат должен записываться в файл;
	\item должен присутствовать замер реального времени для реализаций
алгоритмов;
	\item результат замера должен выводится в виде таблицы.
\end{itemize}

\section{Описание используемых типов данных}

При реализации алгоритмов будут использованы следующие структуры и типы данных:
\begin{itemize}
	\item массив символов для хранения терма;
	\item вещественное число для хранения \textit{TF-IDF} терма;
	\item мьютекс~--- примитив синхронизации.
\end{itemize}

\section{Разработка алгоритмов}

На рисунке~\ref{img:hierarchicalClustering} приведена схема дивизимной иерархической кластеризации.
На рисунке~\ref{img:kmeans} приведена схема алгоритма k-средних.
На рисунке~\ref{img:parallel} представлена схема алгоритма создания потоков для алгоритма дивизимной иерархической кластеризации.
На рисунке~\ref{img:parallelClustering} представлена схема многопоточного алгоритма кластеризации документов.

\includeimage
{hierarchicalClustering} % Имя файла без расширения (файл должен быть расположен в директории inc/img/)
{f} % Обтекание (без обтекания)
{h} % Положение рисунка (см. figure из пакета float)
{1\textwidth} % Ширина рисунка
{Схема дивизимной иерархической кластеризации} % Подпись рисунк
\clearpage


\includeimage
	{kmeans} % Имя файла без расширения (файл должен быть расположен в директории inc/img/)
	{f} % Обтекание (без обтекания)
	{h} % Положение рисунка (см. figure из пакета float)
	{0.9\textwidth} % Ширина рисунка
	{Схема алгоритма k-средних} % Подпись рисунк
 \clearpage


\includeimage
{parallel} % Имя файла без расширения (файл должен быть расположен в директории inc/img/)
{f} % Обтекание (без обтекания)
{h} % Положение рисунка (см. figure из пакета float)
{0.8\textwidth} % Ширина рисунка
{Схема алгоритма создания потоков} % Подпись рисунк
\clearpage

\includeimage
{parallelClustering} % Имя файла без расширения (файл должен быть расположен в директории inc/img/)
{f} % Обтекание (без обтекания)
{h} % Положение рисунка (см. figure из пакета float)
{1\textwidth} % Ширина рисунка
{Схема многопоточного алгоритма кластеризации документов} % Подпись рисунк


\section *{Вывод}

В данном разделе были представлены требования к программному обеспечению, описание используемых типов данных и схемы реализуемых алгоритмов.
