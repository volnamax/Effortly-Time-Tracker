\if 0
Требования к конструкторской части курсовой работы:
➢ Диаграмма проектируемой базы данных;
➢ Описание сущностей проектируемой базы данных;
➢ Описание проектируемых ограничений целостности базы данных;
➢ Описание всех проектируемых процедур/функций/триггеров в формате схемы;
➢ Описание проектируемой ролевой модели на уровне базы данных (у каких ролей какой
доступ и к каким объектам)
\fi

\chapter{Конструкторский раздел}

%todo  
В данном разделе будут представлены требования к программному обеспечению, описание используемых типов данных и схемы реализуемых алгоритмов.


\section{Проектирование базы данных}

Исходя из ER диаграммы, приведенной в предыдущем разделе (см. рисунок~\ref{img:er_rus}), были выделены следующие таблицы :

\begin{itemize}
	\item таблица пользователей (user\_app);
	\item таблица записей списка дел (todo\_node);
	\item таблица проектов (project);
	\item таблица ролей (roles);
	\item таблица групп пользователей (group\_user);
	\item таблица для связи группы с пользователем (group\_member);
	\item таблица таблиц задач (table\_app);
	\item таблица тегов (tag);
	\item таблица задач (task);
	\item таблица для связи задач с тегами (task\_tag).
\end{itemize}


\textbf{Описание сущностей и ограничений целостности в проектируемой базе данных} 

Информация о полях в таблице пользователей (user\_app) представлена в таблице~\ref{tab:user_app}.


%todo add not nul in role_id
\begin{table}[H]
	\centering
	\caption{Описание таблицы пользователей}
	\label{tab:user_app}
	\begin{tabularx}{\textwidth}{|l|l|X|X|}
		\hline
		\textbf{Поле} & \textbf{Тип данных} & \textbf{Ограничение} & \textbf{Описание} \\ \hline
		id\_user  & Целочисленный &  Уникальное, обязательное & Первичный ключ таблицы  \\ \hline
		email   & Символьный &  Уникальное, обязательное, максимальная длина --- 255 символов & Адрес электронной почты пользователя (логин пользователя)\\ \hline
		password  & Символьный &  Обязательное, максимальная длина~---~255 символов &  Хэш пароля пользователя \\ \hline
		user\_name  & Символьный &  Обязательное, максимальная длина~---~255 &  Имя пользователя\\ \hline
		user\_secondname  & Символьный &  Обязательное, максимальная длина~---~255 символов &  Фамилия пользователя\\ \hline
		data\_last\_log\_in & Дата & --- &  Дата последнего входа пользователя\\ \hline
		data\_sign\_in & Дата & --- & Дата регистрации пользователя\\ \hline
		role\_id & Целочисленный & Обязательное & Внешний ключ, для определения роли\\ \hline
	\end{tabularx}
\end{table}

%todo add not null in name
Информация о полях в таблице ролей (roles) представлена в таблице~\ref{tab:roles}.

\begin{table}[H]
	\centering
	\caption{Описание таблицы ролей}
	\label{tab:roles}
	\begin{tabularx}{\textwidth}{|l|l|X|X|}
		\hline
		\textbf{Поле} & \textbf{Тип данных}  & \textbf{Ограничение} & \textbf{Описание} \\ \hline
		id\_role  & Целочисленный   &  Уникальное, обязательное  & Первичный ключ таблицы  \\ \hline
		name   & Символьный  &  Обязательное & Имя роли\\ \hline
	\end{tabularx}
\end{table}


Информация о полях в таблице записи списка дел (todo\_node) представлена в таблице~\ref{tab:todo_node}.

%todo check str or user type add ? 
% todo add user_id not null and in status
\begin{table}[H]
	\centering
	\caption{Описание таблицы записи списка дел}
	\label{tab:todo_node}
	\begin{tabularx}{\textwidth}{|l|l|X|X|}
		\hline
		\textbf{Поле} & \textbf{Тип данных} & \textbf{Ограничение} & \textbf{Описание} \\ \hline
		id\_node  & Целочисленный  &  Уникальное, обязательное   & Первичный ключ таблицы  \\ \hline
		content   & Символьный & Максимальная длина~---~255 символов &  Описание записи\\ \hline
		priority  & Символьный & Максимальная длина~---~255 символов&   Приоритет записи \\ \hline
		status  & Символьный & Максимальная длина~---~255 символов, обязательное  &  Статус записи\\ \hline
		due\_data & Дата & --- & Дата и время срока выполнения записи \\ \hline
		user\_id   &Целочисленный & Обязательное  & Идентификатор пользователя, создавшего запись, внешний ключ   \\ \hline
	\end{tabularx}
\end{table}


Информация о полях в таблице проекта (project) представлена в таблице~\ref{tab:project}.


\begin{table}[H]
	\centering
	\caption{Описание таблицы проекта}
	\label{tab:project}
	\begin{tabularx}{\textwidth}{|l|l|X|X|}
		\hline
		\textbf{Поле} & \textbf{Тип данных} & \textbf{Ограничение} & \textbf{Описание} \\ \hline
		id\_project & Целочисленный &  Уникальное, обязательное   & Первичный ключ таблицы  \\ \hline
		name   & Символьный &  Обязательное, максимальная длина~---~255 символов &  Название проекта  \\ \hline
		description  & Символьный &  Обязательное, максимальная длина~---~255 символов  & Описание  проекта\\ \hline
		user\_id  & Целочисленный & Обязательное    & Идентификатор
		 пользователя, создавшего проект, внешний ключ   \\ \hline
	\end{tabularx}
\end{table}


Информация о полях в таблице таблиц (table\_app) представлена в таблице~\ref{tab:table}.

%todo not null in ststus
\begin{table}[H]
	\centering
	\caption{Описание таблицы таблиц}
	\label{tab:table}
	\begin{tabularx}{\textwidth}{|l|l|X|X|}
		\hline
		\textbf{Поле} & \textbf{Тип данных} & \textbf{Ограничение}  & \textbf{Описание} \\ \hline
		id\_project & Целочисленный  &  Уникальное, обязательное   & Первичный ключ таблицы  \\ \hline
		name   & Символьный   & Максимальная длина~---~255 символов, обязательное &  Название таблицы  \\ \hline
		description  & Символьный   & Максимальная длина~---~255 символов & Описание таблицы\\ \hline
		status  & Символьный  & Максимальная длина~---~255 символов, обязательное & Статус таблицы\\ \hline
		project\_id  & Целочисленный & Обязательное    & Идентификатор проекта к которому относится таблица задач, внешний ключ   \\ \hline
	\end{tabularx}
\end{table}



Информация о полях в таблице задачи (task) представлена в таблице~\ref{tab:task}.

%todo
\begin{table}[H]
	\centering
	\caption{Описание таблицы задачи}
	\label{tab:task}
	\begin{tabularx}{\textwidth}{|l|l|X|X|}
		\hline
		\textbf{Поле} & \textbf{Тип данных} & \textbf{Ограничение} & \textbf{Описание} \\ \hline
		id\_task & Целочисленный    &  Уникальное, обязательное & Первичный ключ таблицы  \\ \hline
		name   & Символьный  & Максимальная длина~---~255 символов, обязательное &  Название задачи  \\ \hline
		
		description  & Символьный & Максимальная длина~---~255 символов & Описание задачи\\ \hline
		status  & Символьный   & Максимальная длина~---~255 символов, обязательное & Статус задачи\\ \hline
		sum\_timer  & Целочисленный & --- & Общее время выполнения задачи\\ \hline
		start\_timer  & Дата & ---& Время начала выполнения задачи \\ \hline
		timer\_add\_task  & Дата & ---& Дата начала выполнения задачи \\ \hline
		time\_end\_task  & Дата & --- & Дата завершения выполнения задачи  \\ \hline
		table\_id  & Целочисленный  & Обязательное   & Идентификатор проекта к которому относится таблица задач, внешний ключ   \\ \hline
	\end{tabularx}
\end{table}



Информация о полях в таблице тегов (tag) представлена в таблице~\ref{tab:tag}.


%todo
\begin{table}[H]
	\centering
	\caption{Описание таблицы тегов}
	\label{tab:tag}
	\begin{tabularx}{\textwidth}{|l|l|X|X|}
		\hline
		\textbf{Поле} & \textbf{Тип данных}& \textbf{Ограничение} & \textbf{Описание} \\ \hline
		id\_tag & Целочисленный  &  Уникальное, обязательное  & Первичный ключ таблицы  \\ \hline
		name   & Символьный & Максимальная длина~---~255 символов, обязательное&  Название тега\\ \hline
		color  & Символьный & Максимальная длина~---~255 символов & цвет тега \\ \hline
		project\_id & Целочисленный    & Обязательное & Идентификатор проекта к которой относится тег, внешний ключ  \\ \hline
	\end{tabularx}
\end{table}



Информация о полях в таблице связи задач с тегами (task\_tag) представлена в таблице~\ref{tab:tasktag}.


%todo
\begin{table}[H]
	\centering
	\caption{Описание таблицы связи задачи и тега}
	\label{tab:tasktag}
	\begin{tabularx}{\textwidth}{|l|l|X|X|}
		\hline
		\textbf{Поле} & \textbf{Тип данных}& \textbf{Ограничение}   & \textbf{Описание} \\ \hline
		id & Целочисленный   &  Уникальное, обязательное  & Первичный ключ таблицы  \\ \hline
		tag\_id   & Целочисленный & Обязательное & Идентификатор тега, внешний ключ \\ \hline
		task\_id & Целочисленный & Обязательное& Идентификатор задачи, внешний ключ\\ \hline
	\end{tabularx}
\end{table}

Информация о полях в таблице групп (group\_user) представлена в таблице~\ref{tab:group}.
%todo
\begin{table}[H]
	\centering
	\caption{Описание таблицы групп}
	\label{tab:group}
	\begin{tabularx}{\textwidth}{|l|l|X|X|}
		\hline
		\textbf{Поле} & \textbf{Тип данных } & \textbf{Ограничение}  & \textbf{Описание} \\ \hline
		id & Целочисленный   &  Уникальное, обязательное   & Первичный ключ таблицы  \\ \hline
		name & Символьный & Максимальная длина~---~255 символов, обязательное & Название группы\\ \hline
		description   & Символьный & Максимальная длина~---~255 символов  & Описание группы\\ \hline
		project\_id & Целочисленный  &  Уникальное, обязательное & Идентификатор проекта, внешний ключ\\ \hline
	\end{tabularx}
\end{table}


Информация о полях в таблице связи группы и пользователя (group\_member) представлена в таблице~\ref{tab:groupmember}.

%todo not null in role
\begin{table}[H]
	\centering
	\caption{Описание таблицы связи группы и пользователя}
	\label{tab:groupmember}
	\begin{tabularx}{\textwidth}{|l|l|X|X|}
		\hline
		\textbf{Поле} & \textbf{Тип данных} & \textbf{Ограничение}   & \textbf{Описание} \\ \hline
		id & Целочисленный  &  Уникальное, обязательное  & Первичный ключ таблицы  \\ \hline
		role & Символьный  & Максимальная длина~---~255 символов, обязательное &  Роль пользователя в группе \\ \hline
		user\_id & Целочисленный & Обязательное& Идентификатор пользователя, внешний ключ\\ \hline
		group\_id & Целочисленный & Обязательное & Идентификатор группы, внешний ключ\\ \hline
	\end{tabularx}
\end{table}






На рисунке~\ref{img:dbDiagram} приведена диаграмма разрабатываемой базы данных.

\includeimage
{dbDiagram} % Имя файла без расширения (файл должен быть расположен в директории inc/img/)
{f} % Обтекание (без обтекания)
{h} % Положение рисунка (см. figure из пакета float)
{1\textwidth} % Ширина рисунка
{Диаграмма разрабатываемой базы данных} % Подпись рисунк





% todo 
\section{Требования к программному обеспечению}

К программе предъявлен ряд требований:
\begin{itemize}
	\item должен присутствовать интерфейс для выбора действий;
	\item считывание данных должно производиться из файла;
	\item результат должен записываться в файл;
	\item должен присутствовать замер реального времени для реализаций
алгоритмов;
	\item результат замера должен выводится в виде таблицы.
\end{itemize}

\section{Описание используемых типов данных}

При реализации алгоритмов будут использованы следующие структуры и типы данных:
\begin{itemize}
	\item массив символов для хранения терма;
	\item вещественное число для хранения \textit{TF-IDF} терма;
	\item мьютекс~---~примитив синхронизации.
\end{itemize}

\section{Разработка алгоритмов}

На рисунке~\ref{img:hierarchicalClustering} приведена схема дивизимной иерархической кластеризации.
На рисунке~\ref{img:kmeans} приведена схема алгоритма k-средних.
На рисунке~\ref{img:parallel} представлена схема алгоритма создания потоков для алгоритма дивизимной иерархической кластеризации.
На рисунке~\ref{img:parallelClustering} представлена схема многопоточного алгоритма кластеризации документов.

\if 0
\includeimage
{hierarchicalClustering} % Имя файла без расширения (файл должен быть расположен в директории inc/img/)
{f} % Обтекание (без обтекания)
{h} % Положение рисунка (см. figure из пакета float)
{1\textwidth} % Ширина рисунка
{Схема дивизимной иерархической кластеризации} % Подпись рисунк
\clearpage

\fi

\section *{Вывод}

В данном разделе были представлены требования к программному обеспечению, описание используемых типов данных и схемы реализуемых алгоритмов.
