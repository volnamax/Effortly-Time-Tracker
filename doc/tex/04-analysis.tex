\chapter{Аналитический раздел}

В данном разделе будет рассмотрено следующее: анализ предметной области, формулировка требований к разрабатываемой базе данных и приложению, формализация и описание информации, анализ существующих баз данных на основе формализации данных, ER-диаграмма, формализация и описание пользователей проектируемого приложения к базе данных, диаграмма вариантов использования.

\section{Анализ предметной области}

Предметная область "эффективное управление временем и задачами" затрагивает процессы планирования, отслеживания и анализа времени, затрачиваемого на выполнение различных задач и проектов. 

Существует множество теорий и методологий, способствующих эффективному управлению временем.
Среди наиболее известных подходов выделяются Матрица Эйзенхауэра, которая помогает приоритизировать задачи по их срочности и важности, метод Помодоро, предлагающий разбивку работы на короткие интервалы с перерывами, и система GTD, нацеленная на снижение стресса за счет переноса задач из головы в внешнюю систему организации.
Современные методологии планирования, такие как Agile, Scrum и Kanban, привносят гибкость и адаптивность в процесс управления проектами и задачами. Они позволяют командам реагировать на изменения и эффективно управлять рабочим процессом, поддерживая постоянную обратную связь и цикличность в деятельности~\cite{matrix, main-book}.

\textbf{Анализ аналогичных решений}

На рынке существует большое количество приложений, которые имеют функциональность замера времени для задач, наиболее известные: clockify, toggl, timecamp, также есть приложения по планирования задач в стиле канбан-досок, например как trello.
Выделим следующие критерии для сравнения:
\begin{itemize}
	\item наличие таймера для задачи;
	\item ограниченность в создание проектов;
	\item наличие аналитики и экспорт ее;
	\item ограниченность в количестве пользователей у проекта;
	\item наличие канбан-досок для удобной организации задач.
\end{itemize}

В таблице~\ref{tab:comparison} представлено сравнение существующих решений для бесплатных версий приложений.

\begin{table}[H]
	\centering
	\caption{Сравнение существующих решений}
	\label{tab:comparison}
	\resizebox{\textwidth}{!}{%
		\begin{tabular}{|l|l|l|l|l|}
			\hline
			\textbf{Критерий} & \textbf{clockify} & \textbf{toggl} & \textbf{timecamp} & \textbf{trello} \\ \hline
			Таймер & Есть & Есть & Есть & Нет \\ \hline
			Количество проектов & Неограниченно & Неограниченно & Ограничено & Ограничено \\ \hline
			Экспорт аналитики & PDF, CSV, Excel & PDF, CSV & PDF & Нет \\ \hline
			Кол-во пользователей & Неограниченно & Ограничено & Ограничено & Ограничено \\ \hline
			Канбан-доски & Нет & Нет & Нет & Есть \\ \hline
		\end{tabular}
	}
\end{table}


Как видно из таблицы~\ref{tab:comparison} ни одно из рассмотренных решений не удовлетворяет всем критериям сравнения, также данные приложения являются зарубежными.
Таким образом мое решение будет актуальным и в отличие от других решений, будет иметь канбан-доски с интегрированным таймером для каждой задачи, облегчая тем самым учет времени и повышая эффективность планирования


\section {Формулировка требований к разрабатываемой базе данных и приложению}

В рамках курсовой работы необходимо разработать базу данных, в которой будет храниться информация о пользователях, их проектах, задачах, карточках, тегах, списках дел и группах.
Также были выделены необходимые требования для приложения.
Приложение должно включать следующие функциональности:
\begin{itemize}
	\item регистрация и аутентификация пользователя;
	\item изменение списка дел, задач проектов и тегов;
	\item поиск задач, проектов и тегов;
	\item изменение групп пользователей;
	\item замер времени для конкретной задачи;
	\item просмотр и сохранение аналитики использования времени, замеренного для определенных задач;
	\item создание и изменение групп пользователей.
\end{itemize}

Необходимо предусмотреть работу с задачами в виде канбан-досок, так как данная функция является отличительной от других приложений.

\section{Формализация и описание информации}
 
Разрабатываемая база данных для приложения по учету и аудиту затраченного времени на определенные задачи, должна содержать информацию о пользователях, их проектах, задачах, карточках, тегах, списках дел и группах.

\includeimage
{er_rus} % Имя файла без расширения (файл должен быть расположен в директории inc/img/)
{f} % Обтекание (без обтекания)
{h} % Положение рисунка (см. figure из пакета float)
{1\textwidth} % Ширина рисунка
{ER диаграмма} % Подпись рисунк
\clearpage

\includeimage
{use-case-new-c} % Имя файла без расширения (файл должен быть расположен в директории inc/img/)
{f} % Обтекание (без обтекания)
{h} % Положение рисунка (см. figure из пакета float)
{1\textwidth} % Ширина рисунка
{Диаграмма использования} % Подпись рисунк
\clearpage



\section *{Вывод}
В данном разделе были рассмотрены многопоточность, алгоритмы классификации полнотекстовых документов, алгоритмы классификации без учителя, иерархические алгоритмы, алгоритм дивизимной иерархической кластеризации, алгоритм дивизимной иерархической кластеризации и алгоритм k-средних.
