\chapter{Аналитический раздел}

В данном разделе будет рассмотрено следующее: анализ предметной области, формулировка требований к разрабатываемой базе данных и приложению, формализация и описание информации, анализ существующих баз данных на основе формализации данных, ER-диаграмма, формализация и описание пользователей проектируемого приложения к базе данных, диаграмма вариантов использования.

\section{Анализ предметной области}

Предметная область "эффективное управление временем и задачами" затрагивает процессы планирования, отслеживания и анализа времени, затрачиваемого на выполнение различных задач и проектов. 

Существует множество теорий и методологий, способствующих эффективному управлению временем.
Среди наиболее известных подходов выделяются Матрица Эйзенхауэра, которая помогает приоритизировать задачи по их срочности и важности, метод Помодоро, предлагающий разбивку работы на короткие интервалы с перерывами, и система GTD, нацеленная на снижение стресса за счет переноса задач из головы в внешнюю систему организации.
Современные методологии планирования, такие как Agile, Scrum и Kanban, привносят гибкость и адаптивность в процесс управления проектами и задачами. Они позволяют командам реагировать на изменения и эффективно управлять рабочим процессом, поддерживая постоянную обратную связь и цикличность в деятельности~\cite{matrix, main-book}.

\textbf{Анализ аналогичных решений}

На рынке существует большое количество приложений, которые имеют функциональность замера времени для задач, наиболее известные: clockify, toggl, timecamp, также есть приложения по планирования задач в стиле канбан-досок, например как trello.
Выделим следующие критерии для сравнения:
\begin{itemize}
	\item наличие таймера для задачи;
	\item ограниченность в создание проектов;
	\item наличие аналитики и экспорт ее;
	\item ограниченность в количестве пользователей у проекта;
	\item наличие канбан-досок для удобной организации задач.
\end{itemize}

В таблице~\ref{tab:comparison} представлено сравнение существующих решений для бесплатных версий приложений.

\begin{table}[H]
	\centering
	\caption{Сравнение существующих решений}
	\label{tab:comparison}
	\resizebox{\textwidth}{!}{%
		\begin{tabular}{|l|l|l|l|l|}
			\hline
			\textbf{Критерий} & \textbf{clockify} & \textbf{toggl} & \textbf{timecamp} & \textbf{trello} \\ \hline
			Таймер & Есть & Есть & Есть & Нет \\ \hline
			Количество проектов & Неограниченно & Неограниченно & Ограничено & Ограничено \\ \hline
			Экспорт аналитики & PDF, CSV, Excel & PDF, CSV & PDF & Нет \\ \hline
			Кол-во пользователей & Неограниченно & Ограничено & Ограничено & Ограничено \\ \hline
			Канбан-доски & Нет & Нет & Нет & Есть \\ \hline
		\end{tabular}
	}
\end{table}


Как видно из таблицы~\ref{tab:comparison} ни одно из рассмотренных решений не удовлетворяет всем критериям сравнения, также данные приложения являются зарубежными.
Таким образом мое решение будет актуальным и в отличие от других решений, будет иметь канбан-доски с интегрированным таймером для каждой задачи, облегчая тем самым учет времени и повышая эффективность планирования


\section {Формулировка требований к разрабатываемой базе данных и приложению}

В рамках курсовой работы необходимо разработать базу данных, в которой будет храниться информация о пользователях, их проектах, задачах, карточках, тегах, списках дел и группах.
Также были выделены необходимые требования для приложения.
Приложение должно включать следующие функциональности:
\begin{itemize}
	\item регистрация и аутентификация пользователя;
	\item изменение списка дел, задач проектов и тегов;
	\item поиск задач, проектов и тегов;
	\item изменение групп пользователей;
	\item замер времени для конкретной задачи;
	\item просмотр и сохранение аналитики использования времени, замеренного для определенных задач;
	\item создание и изменение групп пользователей.
\end{itemize}

Необходимо предусмотреть работу с задачами в виде канбан-досок, так как данная функция является отличительной от других приложений.

\section{Формализация и описание информации}
 
Разрабатываемая база данных для приложения по учету и аудиту затраченного времени на определенные задачи, должна содержать информацию о пользователях, их проектах, задачах, карточках, тегах, списках дел и группах. 

В таблице~\ref{tab:er} представлены необходимые таблицы и информация, которая должна содержаться в этой таблице для разрабатываемой базы данных.

\begin{table}[H]
	\centering
	\caption{Таблиц и их информаций}
	\label{tab:er}
	\begin{tabularx}{\textwidth}{|l|X|}
		\hline
		\textbf{Таблица} & \textbf{Информация} \\ \hline
		Пользователь      & id, имя, фамилия, роль, почта, дата последнего входа и дата регистрации    \\ \hline
		Проект            & id, название, описание  \\ \hline
		Задача            & id, название, статус, описание, дата начала и завершения задачи, сумма времени таймера, \\ \hline
		Группа            & id, название, описание  \\ \hline
		Тег               & id, название, цвет  \\ \hline
		Список дел        & id, статус, приоритет, дата завершения, контент \\ \hline
	\end{tabularx}
\end{table}
\clearpage

На рисунке~\ref{img:er_rus} приведена ER диаграмма в нотации Чена.
\includeimage
{er_rus} % Имя файла без расширения (файл должен быть расположен в директории inc/img/)
{f} % Обтекание (без обтекания)
{h} % Положение рисунка (см. figure из пакета float)
{1\textwidth} % Ширина рисунка
{ER диаграмма} % Подпись рисунк

\clearpage 
\section{Формализация и описание пользователей проектируемого приложения к базе данных}

Для взаимодействия с приложением по учету и аудиту времени, было выделено три роли пользователей: неавторизированный, авторизированный и администратор.

В таблице~\ref{tab:functionality} представлена функциональность для администратора, неавторизированного и авторизированного пользователя. 
Также в таблице~\ref{tab:functionality} используются сокращения: 1 означает пользователь администратор, 2~---~неавторизированный пользователь, 3~---~авторизированный пользователь.

\begin{table}[h!]
	\caption{Функциональность администратора, неавторизированного и авторизированного пользователя}
	\label{tab:functionality}
	\centering
	
	\begin{tabular}{|l|l|l|l|}
		\hline
		\textbf{Функциональность} & \textbf{1} & \textbf{2} & \textbf{3} \\ \hline
		Зарегистрироваться и войти в систему & + & + & + \\ \hline
		Изменить список дел &  & + & + \\ \hline
		Создать и изменить личный проект, задачи и теги &  & + & + \\ \hline
		Провести поиск задач, проектов и тегов &  & + & + \\ \hline
		Замерить время для конкретной задачи  &  & + & + \\ \hline
		Просмотреть и сохранить аналитику &  & + & + \\ \hline
		Создать группу &  &  & + \\ \hline
		Изменить группу (если владелец) &  &  & + \\ \hline
		Просмотреть и сохранить аналитику (группы) & + &   & + \\ \hline
		Изменить группы пользователей & + &  &  \\ \hline
		Удалить пользователя из системы & + &  &  \\ \hline
	\end{tabular}
	
\end{table}




\clearpage
\section{Диаграмма вариантов использования}

На рисунке~\ref{img:use-case-new-c} приведена диаграмма вариантов использования.
\includeimage
{use-case-new-c} % Имя файла без расширения (файл должен быть расположен в директории inc/img/)
{f} % Обтекание (без обтекания)
{h} % Положение рисунка (см. figure из пакета float)
{1\textwidth} % Ширина рисунка
{Диаграмма использования} % Подпись рисунк

\clearpage 

%todo

\section{Анализ существующих баз данных на основе формализации данных}

В контексте темы курсовой работы и проведенной формализации задачи, данных и пользователей, была выбрана реляционная модель данных.

Реляционная модель данных представляет собой подход к управлению данными, где все данные хранятся в таблицах (также называемых "отношениями"), состоящих из строк и столбцов. Каждая строка таблицы представляет собой запись с уникальным идентификатором (ключом), а каждый столбец — атрибутом данных. Реляционные базы данных обеспечивают гибкость в обработке данных, возможность выполнения сложных запросов и обеспечения целостности данных через систему отношений между таблицами.

\textbf{SQLite}

SQLite — это встраиваемая, высоконадежная библиотека СУБД, которая предоставляет реляционную базу данных в виде файла. Её главное преимущество — не требует отдельного серверного процесса или системы управления, что делает SQLite идеальным выбором для мобильных приложений, приложений для настольных компьютеров и небольших веб-проектов, где требуется простота без жертвы функциональности. Она поддерживает основные операции SQL и отличается малым размером, быстродействием и надежностью.

\textbf{PostgreSQL}

PostgreSQL — это мощная, открытая и полностью бесплатная объектно-реляционная система управления базами данных (ОРСУБД). Она предоставляет расширенные возможности SQL, включая сложные запросы, внешние ключи, триггеры, представления, транзакции, индексирование полнотекстового поиска и сохраненные процедуры. PostgreSQL является хорошим выбором для крупных приложений и систем, требующих высокой надежности, масштабируемости и гибкости.

\textbf{MySQL}

MySQL — это популярная открытая реляционная система управления базами данных. Она широко используется в веб-разработке и является частью стека LAMP (Linux, Apache, MySQL, PHP/Python/Perl). MySQL предлагает высокую производительность, надежность и простоту в использовании, а также поддерживает широкий спектр операционных систем. Эта СУБД идеально подходит для веб-сайтов и приложений среднего и большого размера, требующих эффективного управления данными.


Эти СУБД выбраны из-за их популярности, мощности, гибкости и разнообразия функций, которые они предлагают для разработчиков мобильных приложений, веб-приложений и крупных систем. 

\begin{table}[h]
	\caption{Сравнение реляционных СУБД}
	\label{tab:database_comparison}
	\begin{tabular}{|l|p{3cm}|p{3cm}|p{3cm}|}
		\hline
		\textbf{Критерий} & \textbf{SQLite} & \textbf{PostgreSQL} & \textbf{MySQL} \\
		\hline
		Лицензия & Public Domain \cite{sqlite} & PostgreSQL License, открытая \cite{postgresql} & GPL для бесплатной версии или коммерческая лицензия \cite{mysql} \\
		\hline
		Тип хранения & Встраиваемая & Клиент-сервер & Клиент-сервер \\
		\hline
		Поддержка JSON & Частичная & Полная & Полная \\
		\hline
		Репликация & Поддержка через сторонние инструменты & Нативная поддержка & Нативная поддержка \\
		\hline
		Полнотекстовый поиск & Да & Да & Да \\
		\hline
	\end{tabular}
\end{table}
\textbf{Источники:}

SQLite: \href{https://www.sqlite.org/index.html}{Официальный сайт SQLite}.

PostgreSQL: \href{https://www.postgresql.org/}{Официальный сайт PostgreSQL}.

MySQL: \href{https://www.mysql.com/}{Официальный сайт MySQL}.



SQLite был выбран как целевая СУБД для разработки мобильного приложения на Kotlin по нескольким ключевым причинам, учитывая специфику задачи, данные и пользователей:

Встраиваемость и Легковесность: SQLite — это встраиваемая база данных, что означает, что она легко интегрируется непосредственно в мобильное приложение без необходимости во внешнем сервере баз данных. Это делает SQLite особенно подходящим для мобильных приложений, где необходимо минимизировать использование ресурсов и обеспечить высокую производительность.

Независимость и Портативность: SQLite хранит всю базу данных в одном файле, что облегчает передачу, резервное копирование и синхронизацию данных между устройствами. Это важно для мобильных приложений, которые могут работать в автономном режиме или в условиях ненадежного сетевого соединения.

Простота Управления: SQLite не требует настройки, установки и управления сервером баз данных, что снижает затраты на поддержку и управление приложением. Это делает его идеальным выбором для малых и средних проектов, где может не хватать ресурсов для обслуживания сложной инфраструктуры баз данных.

Поддержка Транзакций и ACID: SQLite полностью поддерживает транзакции и соответствует принципам ACID (атомарность, согласованность, изоляция, долговечность), что гарантирует надежность и целостность данных в приложении.

Широкая Поддержка и Сообщество: SQLite имеет обширную документацию и поддерживается большим сообществом разработчиков. Это обеспечивает доступность ресурсов, библиотек и инструментов для разработки, тестирования и отладки приложений.

Гибкость: Несмотря на свою простоту, SQLite предлагает достаточный набор функций SQL, включая поддержку подзапросов, внешних ключей, триггеров и представлений, что позволяет эффективно управлять данными в мобильном приложении.

В совокупности, эти преимущества делают SQLite оптимальным выбором для мобильных приложений, где требуется надежное, эффективное и удобное в обслуживании решение для хранения данных.


\section *{Вывод}
В данном разделе было рассмотрено следующее анализ предметной области, формулировка требований к разрабатываемой базе данных и приложению, формализация и описание информации, анализ существующих баз данных на основе формализации данных, ER-диаграмма, формализация и описание пользователей проектируемого приложения к базе данных, диаграмма вариантов использования.